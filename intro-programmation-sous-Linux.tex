% fichier Exposé-MontPellibre-juil2020/intro-programmation-sous-Linux.tex
\documentclass[xcolor=svgnames,final,smaller,a4]{beamer}
\usepackage{relsize}
\usepackage{luacode}
\usepackage{xcolor}
\usepackage{alltt}
\usepackage{wasysym}
\usepackage{hyperref}
\usepackage{newunicodechar}

% see also http://www.sascha-frank.com/Arrow/latex-arrows.html
% and http://tug.ctan.org/info/symbols/comprehensive/symbols-a4.pdf
% and https://ctan.math.illinois.edu/macros/latex/contrib/newunicodechar/newunicodechar.pdf
%%%% keep in order
%U+21A6 RIGHTWARDS ARROW FROM BAR
\newunicodechar{↦}{$\mapsto$}
%U+21B3 DOWNWARDS ARROW WITH TIP RIGHTWARDS
\newunicodechar{↳}{\rotatebox[origin=c]{180}{$\Lsh$}}
%U+2208 ELEMENT OF
\newunicodechar{∈}{$\in$}
% U+00AB LEFT-POINTING DOUBLE ANGLE QUOTATION MARK
\newunicodechar{«}{\guillemotleft}
% U+00BB RIGHT-POINTING DOUBLE ANGLE QUOTATION MARK
\newunicodechar{»}{\guillemotright}
% U+00B1 PLUS-MINUS SIGN
\newunicodechar{±}{$\pm$}
% U+00B5 MICRO SIGN
\newunicodechar{µ}{$\mu$}



\hypersetup{
  colorlinks   = true, %Colours links instead of ugly boxes
  urlcolor     = NavyBlue, %Colour for external hyperlinks
  linkcolor    = DarkGreen, %Colour of internal links
  citecolor   = DarkMagenta, %Colour of citations
  frenchlinks = true,
}

\usetheme{Montpellier}



\title{\textsc{Introduction à la Programmation} \\
(sous Linux)}
\author[B.Starynkevitch]{Basile \textsc{Starynkévitch} - \href{http://starynkevitch.net/Basile/}{\texttt{starynkevitch.net/Basile}}\\ \href{mailto:basile@starynkevitch.net}{\color{blue}{\texttt{basile@starynkevitch.net}}}
} %
\institute{MontPellibre (Montpellier)}
\date{été 2020}

\begin{document}
 \begin{luacode*}
   local gitpip=io.popen("git log --no-color --format=oneline -1 --abbrev=16 --abbrev-commit -q | cut -d' ' -f1")
   gitid=gitpip:read()
   gitpip:close()
 \end{luacode*}
 \newcommand{\mygitid}{\luadirect{tex.print(gitid)}}

%{% open a Local TeX Group
%\setbeamertemplate{sidebar}{}
 \begin{frame}
   
   
   \begin{relsize}{-1.5}
        \titlepage
        \textcolor{brown}{{\large \textbf{Les opinions me sont personnelles}} }
        
        \begin{center}
          git \texttt{\mygitid} ~ 
          \href{https://montpellibre.fr/spip.php?article4875}{https://montpellibre.fr/spip.php?article4875}
        \end{center}
   \end{relsize}
\end{frame}
%}% end Local TeX Group

 \begin{frame}
    \frametitle{Licence}
    
   Ces transparents sont sous license \href{https://creativecommons.org/licenses/by-sa/4.0/}{\includegraphics[scale=0.75]{CC-BY-SA-4}} \relsize{-1}{(CC-BY-SA-4)}
 \end{frame}
 
 \begin{frame}
    \frametitle{Plan}
    
   \tableofcontents
 \end{frame}


 \AtBeginSection[]{
   \begin{frame}{Sommaire}
     \small \tableofcontents[currentsection, hideothersubsections]
   \end{frame}
 }
 
\section{Introduction}


\begin{frame}
  \frametitle{Introduction}
  \framesubtitle{Qu'est ce que l'information}

  \begin{block}{un bit}

    \textbf{Quantité ``élémentaire'' d'information}. \textcolor{red}{Le jeu de pile ou face} transmet approximativement \textcolor{purple}{\textbf{un bit}}, car il y a $2 = 2^1$ possibilités
    
  \end{block}

  Remarques:
  \begin{itemize}
  \item on a fait \textbf{abstraction} des autres possibilités (la pièce perdue dans le caniveau ...)
  \item on a fait une \textbf{simplification} et une \textbf{modélisation} de la réalité.
    \item on a évidemment $log_2 ~ 2 = 1$ car $2 = 2^1$
  \end{itemize}

  Mais l'\emph{abstraction}, la \emph{simplification}, la \emph{modélisation} sont \textbf{au c{\oe}ur de l'activité de programmation}.
  
\end{frame}

\begin{frame}
  \frametitle{Introduction}
  \framesubtitle{Combien de bits transmis au jeu de dés?}
  
  \textbf{un dé a 6 faces}, donc plus de 2 et moins de 3 bits transmis, puisque
  $4 = 2^2 < 6 < 2^3 = 8$

  \vspace{1cm}

  \emph{``informatiquement''} on a transmis $log_2~ 6$ bits, donc $\approx 2,58496$ bits

  \vspace{1cm}

  Q: \textit{combien de bits pour le jeu de la roulette?} (36 cas)
  
\end{frame}

\begin{frame}
  \frametitle{Introduction}
  \framesubtitle{Que faire avec un bit}

  
  \textbf{Représenter toutes choses à deux possibilités}
  
  \begin{itemize}

  \item valeur de vérité en logique : \textbf{vrai} ou \textbf{faux}

  \item comparaison ($<$ ou $>$) entre deux grandeurs (longueur, tension électrique, etc...)
  \item chiffre binaire

    \item signe $+$ ou $-$
  \end{itemize}

  \vspace{1cm}

  \textbf{Distinction entre \textcolor{red}{chiffre} et \textcolor{red}{nombre}}
\end{frame}

\begin{frame}
  \frametitle{Introduction}
  \framesubtitle{Comment représenter \emph{physiquement} un bit}

  Utiliser, ou simplifier, par un \textbf{phénomène physique à \textcolor{red}{deux états}}

  \begin{itemize}
  \item interrupteur marche/arrêt (donc tension électrique: $\approx 0V$ vs $1V$ à $3V$)
  \item pendule mécanique (à gauche ou à droite), horlogerie (Babbage)
    \item onde sonore
    \item magnétisation (tambour, disque dur)
  \item tubes à vide (ENIAC), transistors, circuits intégrés
  \item etc...
  \end{itemize}
  
  \vspace{2mm}
 
  NB. Certaines technologies {\relsize{-1}{(mémoire Flash, SSD, Clefs USB)}} représentent deux bits par 4 états. 
  Mais le matériel est imparfait.
\end{frame}


\begin{frame}
  \frametitle{Introduction}
  \framesubtitle{Notation décimale, binaire,  octale, hexadécimale des nombres entiers}


  \begin{itemize}
    \item
    \textit
   {cent vingt trois} = $123$ \textcolor{red}{en décimal}
  car $1 \times 10^2 + 2 \times 10^1 + 3 \times 10^0$

  \item
  \textit{dix} = $1010_b$ ou \texttt{0b1010}
  \textcolor{red}{en binaire}
  car $1  \times 2^3 + 0 \times 2^2 + 1 \times 2^1 + 0 \times 2^0$

  \item
  \textit{douze} = $14_o$ ou \texttt{014} ou \texttt{0o14}
  \textcolor{red}{en octal}
  car $1  \times 8^1 + 4 \times 8^0$

  \item \textit{vingt} =  $14_h$ ou \texttt{0x14}
  \textcolor{red}{en hexadécimal}
  car $1  \times 16^1 + 4 \times 16^0$
 
  \end{itemize}
  
  \vspace{1cm}
  Remarque: un chiffre octal correspond à 3 bits; un chiffre hexadécimal correspond à 4 bits.
\end{frame}
\begin{frame}
  \frametitle{Introduction}
  \framesubtitle{Octets}

Un  \textcolor{red}{\textbf{octet}} (anglais: \textit{byte}) contient huit bits.

\begin{itemize}
\item petit entier entre $0$ et $2^8-1$ donc $255$

\item entier signé entre -128 ($-2^7$) et +127 ($2^7-1$)

\item codage de caractères (lettres, chiffres, ponctuation)
\href{https://en.wikipedia.org/wiki/ASCII}{en.wikipedia.org/wiki/ASCII}

\item codage universel Unicode UTF-8 sur un à quatre octets
\href{https://en.wikipedia.org/wiki/UTF-8}{\texttt{en.wikipedia.org/wiki/UTF-8}}
et \href{https://utf8everywhere.org/}{\texttt{utf8everywhere.org}}

\end{itemize}
\end{frame}



\begin{frame}
  \frametitle{Introduction}
  \framesubtitle{Circuits logiques combinatoires}

Voir \href{https://www.positron-libre.com/cours/electronique/logique-combinatoire/}{\texttt{www.positron-libre.com/cours/electronique/logique-combinatoire/}}

\vspace{5cm}

Les portes \textbf{NAND} ou \textbf{NOR} sont \textit{universelles}

\end{frame}


\begin{frame}
  \frametitle{Introduction}
  \framesubtitle{Circuits logiques circulaires}

La bascule flip-flop (verrou RS) à deux portes  \textbf{NOR} têtes-bêches. 
\vspace{5cm}

Voir \href{http://www.paturage.be/electro/inforauto/portes/bascule.html}{www.paturage.be/electro/inforauto/portes/bascule.html}

\end{frame}



\begin{frame}
  \frametitle{Introduction}
  \framesubtitle{Combinaisons de portes logiques}

En combinant beaucoup (millions) de portes logiques on produit les microprocesseurs actuels (milliard de transistors, 
AMD ThreadRipper Ryzen)

\vspace{0.5cm}

 \href{https://cdn.arstechnica.net/wp-content/uploads/2017/02/ryzen-die.jpg}{\texttt{cdn.arstechnica.net/wp-content/uploads/2017/02/ryzen-die.jpg}}

 \vspace{0.5cm}
 \includegraphics[width=0.7\textwidth]{ryzen-die.jpg}

\end{frame}

\begin{frame}
  \frametitle{Introduction}
  \framesubtitle{Ordres de grandeurs économiques en 2020}

\begin{itemize}
\item investissement pour la R\&D, la fabrication, l'usine pour un microprocesseur haut de gamme: dix milliards de US\$
\item nombre de transistors par puce: milliard (sur quelques centimètres carrés)
\item prix de vente (après test): environ mille dollars / pièce (milliers)
\item taux de rendement: quelques pourcents, la plupart des puces sont défaillantes
\item taux de panne par porte : moins d'une défaillance par heure
\item consommation énergetique du numérique: quelques pourcents de la production électrique française
\item prix de vente d'un ordinateur: centaines d'euros.
\end{itemize}
\end{frame}

\begin{frame}
  \frametitle{Introduction}
  \framesubtitle{Ordres de grandeurs technologiques en 2020}
 
\begin{itemize}
\item 8 c{\oe}urs de processeur à 3GHz
\item mémoire cache: 16 Mo (accès 10ns, bande passante 0,5To/s)
\item taille mémoire RAM d'un ordinateur : 32 Goctets
\item temps d'accès RAM: 200 ns
\item bande passante RAM: centaine Mo/seconde
\item taille disque SSD : un téra-octets
\item puissance thermique dissipée en charge: 250W
\item temps d'accès SSD: 50 microsecondes pour un bloc de 4Ko.
\end{itemize}

\end{frame}

\begin{frame}

  \frametitle{Introduction}
  \framesubtitle{Vue interne d'un ordinateur de bureau en 2020}

\href{https://images.app.goo.gl/BMQxP7c4Zcyrh3XD7}{images.app.goo.gl/BMQxP7c4Zcyrh3XD7}

\includegraphics[width=0.66\textwidth]{interieur-ordinateur.png}
\end{frame}

%%%%%%%%%%%%%%%%%%%%%%%%%%%%%%%%%%%%%%%%%%%%%%%%%%%%%%%%%%%%%%%%%%%%%
\end{document}
%%%%%%%%%%%%%%%%%%%%%%%%%%%%%%%%%%%a%%%%%%%%%%%%%%%%%%%%%%%%%%%%%
%% Local Variables: ;;
%% compile-command: "./build.sh" ;;
%% End: ;;
%%%%%%%%%%%%%%%%%%%%%%%%%%%%%%%%%%%%%%%%%%%%%%%%%%%%%%%%%%%%%%%%
