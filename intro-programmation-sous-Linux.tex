% fichier Exposé-MontPellibre-juil2020/intro-programmation-sous-Linux.tex
\documentclass[xcolor=svgnames,final,smaller,a4]{beamer}
\usepackage{relsize}
\usepackage{luacode}
\usepackage{xcolor}
\usepackage{alltt}
\usepackage{wasysym}
\usepackage{hyperref}
\usepackage{newunicodechar}

% see also http://www.sascha-frank.com/Arrow/latex-arrows.html
% and http://tug.ctan.org/info/symbols/comprehensive/symbols-a4.pdf
% and https://ctan.math.illinois.edu/macros/latex/contrib/newunicodechar/newunicodechar.pdf
%%%% keep in order
%U+21A6 RIGHTWARDS ARROW FROM BAR
\newunicodechar{↦}{$\mapsto$}
%U+21B3 DOWNWARDS ARROW WITH TIP RIGHTWARDS
\newunicodechar{↳}{\rotatebox[origin=c]{180}{$\Lsh$}}
%U+2208 ELEMENT OF
\newunicodechar{∈}{$\in$}
% U+00AB LEFT-POINTING DOUBLE ANGLE QUOTATION MARK
\newunicodechar{«}{\guillemotleft}
% U+00BB RIGHT-POINTING DOUBLE ANGLE QUOTATION MARK
\newunicodechar{»}{\guillemotright}
% U+00B1 PLUS-MINUS SIGN
\newunicodechar{±}{$\pm$}
% U+00B5 MICRO SIGN
\newunicodechar{µ}{$\mu$}



\hypersetup{
  colorlinks   = true, %Colours links instead of ugly boxes
  urlcolor     = NavyBlue, %Colour for external hyperlinks
  linkcolor    = DarkGreen, %Colour of internal links
  citecolor   = DarkMagenta, %Colour of citations
  frenchlinks = true,
}

\usetheme{Montpellier}



\title{\textsc{Introduction à la Programmation} \\
(sous Linux)}
\author[B.Starynkevitch]{Basile \textsc{Starynkévitch} - \href{http://starynkevitch.net/Basile/}{\texttt{starynkevitch.net/Basile}}\\ \href{mailto:basile@starynkevitch.net}{\color{blue}{\texttt{basile@starynkevitch.net}}}
} %
\institute{MontPellibre (Montpellier)}
\date{été 2020}

\begin{document}
 \begin{luacode*}
   local gitpip=io.popen("git log --no-color --format=oneline -1 --abbrev=16 --abbrev-commit -q | cut -d' ' -f1")
   gitid=gitpip:read()
   gitpip:close()
 \end{luacode*}
 \newcommand{\mygitid}{\luadirect{tex.print(gitid)}}

%{% open a Local TeX Group
%\setbeamertemplate{sidebar}{}
 \begin{frame}
   
   
   \begin{relsize}{-1.5}
        \titlepage
        \textcolor{brown}{{\large \textbf{Les opinions me sont personnelles}} }
        
        \begin{center}
          git \texttt{\mygitid} ~ 
          \href{https://montpellibre.fr/spip.php?article4875}{https://montpellibre.fr/spip.php?article4875}
        \end{center}
   \end{relsize}
\end{frame}
%}% end Local TeX Group

 \begin{frame}
    \frametitle{Licence}
    
   Ces transparents sont sous license \href{https://creativecommons.org/licenses/by-sa/4.0/}{\includegraphics[scale=0.75]{CC-BY-SA-4}} \relsize{-1}{(CC-BY-SA-4)}
 \end{frame}
 
\section{Introduction}


\begin{frame}
  \frametitle{Introduction}
  \framesubtitle{Qu'est ce que l'information}

  \begin{block}{un bit}

    \textbf{Quantité ``élémentaire'' d'information}. \textcolor{red}{Le jeu de pile ou face} transmet approximativement \textcolor{purple}{\textbf{un bit}}, car il y a $2 = 2^1$ possibilités
    
  \end{block}

  Remarques:
  \begin{itemize}
  \item on a fait \textbf{abstraction} des autres possibilités (la pièce perdue dans le caniveau ...)
  \item on a fait une \textbf{simplification} et une \textbf{modélisation} de la réalité.
    \item on a évidemment $log_2 ~ 2 = 1$ car $2 = 2^1$
  \end{itemize}

  Mais l'\emph{abstraction}, la \emph{simplification}, la \emph{modélisation} sont \textbf{au c{\oe}ur de l'activité de programmation}.
  
\end{frame}

\begin{frame}
  \frametitle{Introduction}
  \framesubtitle{Combien de bits transmis au jeu de dés?}
  
  \textbf{un dé a 6 faces}, donc plus de 2 et moins de 3 bits transmis, puisque
  $4 = 2^2 < 6 < 2^3 = 8$

  \vspace{1cm}

  \emph{``informatiquement''} on a transmis $log_2~ 6$ bits, donc $\approx 2,58496$ bits

  \vspace{1cm}

  Q: \textit{combien de bits pour le jeu de la roulette?} (36 cas)
  
\end{frame}

\begin{frame}
  \frametitle{Introduction}
  \framesubtitle{Que faire avec un bit}

  
  \textbf{Représenter toutes choses à deux possibilités}
  
  \begin{itemize}

  \item valeur de vérité en logique : \textbf{vrai} ou \textbf{faux}

  \item comparaison ($<$ ou $>$) entre deux grandeurs (longueur, tension électrique, etc...)
  \item chiffre binaire

    \item signe $+$ ou $-$
  \end{itemize}

  \vspace{1cm}

  \textbf{Distinction entre \textcolor{red}{chiffre} et \textcolor{red}{nombre}}
\end{frame}

\begin{frame}
  \frametitle{Introduction}
  \framesubtitle{Comment représenter \emph{physiquement} un bit}

  Utiliser, ou simplifier, par un \textbf{phénomène physique à \textcolor{red}{deux états}}

  \begin{itemize}
  \item interrupteur marche/arrêt (donc tension électrique: $\approx 0V$ vs $1V$ à $3V$)
  \item pendule mécanique (à gauche ou à droite), horlogerie (Babbage)
    \item onde sonore
    \item magnétisation (tambour, disque dur)
  \item tubes à vide (ENIAC), transistors, circuits intégrés
  \item etc...
  \end{itemize}
  
\end{frame}
\end{document}
%%%%%%%%%%%%%%%%%%%%%%%%%%%%%%%%%%%%%%%%%%%%%%%%%%%%%%%%%%%%%%%%
%% Local Variables: ;;
%% compile-command: "./build.sh" ;;
%% End: ;;
%%%%%%%%%%%%%%%%%%%%%%%%%%%%%%%%%%%%%%%%%%%%%%%%%%%%%%%%%%%%%%%%
